\documentclass[twoside]{article}

\usepackage{ustj}

\newcommand{\authorname}{N. E. Davis}
\newcommand{\authorpatp}{\patp{lagrev-nocfep}}
\newcommand{\affiliation}{Urbit Foundation}

%  Make first page footer:
\fancypagestyle{firststyle}{%
\fancyhf{}% Clear header/footer
\fancyhead{}
\fancyfoot[L]{{\footnotesize
              %% We toggle between these:
              % Manuscript submitted for review.\\
            %   {\it Urbit Systems Technical Journal} I:1 (2024):  1. \\
            %   ~ \\
            %   Address editorial correspondence to \authorpatp.
              }}
}
%  Arrange subsequent pages:
\fancyhf{}
\fancyhead[LE]{{\urbitfont Urbit Systems Technical Journal}}
\fancyhead[RO]{Letter from the Editor}
\fancyfoot[LE,RO]{\thepage}

%%MANUSCRIPT
\title{From the Editor}
\author{\authorname~\authorpatp \\ \affiliation}
\date{}

\begin{document}

%\maketitle
\thispagestyle{firststyle}

% We will adjust page numbering in final editing.
\pagenumbering{arabic}
\setcounter{page}{1}

Welcome to the second issue of the \emph{Urbit Systems Technical Journal:  The Journal of Solid-State Computing}.  Sequels rarely live up to their forebears.  Artists always struggle with their sophomore album.  But Urbit is a deep well with always more water to draw, and this issue of \ssc{ustj} showcases more of the ramifications of truly solid-state computing.

Since our first issue laid out aspects of the development of Urbit's kernel and userspace, the core development team has been hard at work on accelerating Urbit's processing speed, both the evaluation of Nock and the communications network.  As we wrote in the last issue, ``There are deep and true things that can be said about the platonic Urbit, the diamond Urbit, but much of what we are working through now is a contingent Urbit, feeling our way towards zero kelvin.''  It's never quiet on Mars, but it is slowly cooling.

To the extent modest themes have emerged in this issue, the nature of identity and the structure of Azimuth as a public-key infrastructure are explored; aspects of Clay and Eyre as vanes of Arvo are expounded upon; and several historically important documents have been restored from gists and blog posts to a more permanent record.

We trust that the reader will find this issue as illuminating as its predecessor and its successors.  Computer science happens in the trenches.  \tombstone{}
\end{document}

% Issue 2:  When I first encountered Urbit, it was like Mr Electrico touching twelve-year-old Ray Bradbury with a charged sword and commanding him to “live forever!”
% Exit the only game in town.
